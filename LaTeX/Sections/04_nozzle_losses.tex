\section{Nozzle losses}
\label{sec:nozzle_losses}

In order to calculate and evaluate the nozzle losses, further modifications and calculations were added to the model previously presented in \autoref{sec:modelling}.
In particular 2D, throat erosion and boundary layer losses were considered.

\subsection{Losses calculations}  

Each loss term has been calculated as follows:
\begin{itemize}
    \item \textbf{2D losses}: for a parabolic Rao nozzle this loss can be computed in a similar way as a conical nozzle by applying \autoref{eq:2D_loss}
    \begin{equation}
        \lambda = \frac{1}{2} \left[ 1 + \cos \left( \frac{\delta + \theta_e}{2}\right)\right]
        \label{eq:2D_loss}
    \end{equation}
    where $\delta$ is the cone angle of an fictitious conical nozzle with the same divergent length and area ratio \cite{sutton}.
    \item \textbf{Throat erosion losses:} this effect is due to the increasing throat area over time whose behavior can be obtained by considering a constant erosion rate for simplicity. Since this loss is time dependent it needs to be considered inside the dynamic model of the system (\autoref{subsec:dynamics}).
    \begin{equation}
        D_t^{(i+1)} = D_t^{(i)} + 2 E_t \Delta t
    \end{equation}

    Usually the erosion rate is calculated through experimental measurements of the propulsion system, in this case a suitable erosion rate has been searched for in literature. Due to the smallness of the system, no acceptable rates were found, therefore an increase of 2\% of the initial throat radius over the entire burn was assumed.
    This gives an erosion rate of about $1.540 \cdot 10^{-2}$ $\mu$m/s. \cite{sutton}\cite{tesi_malesia}
    \item \textbf{Boundary layer losses:} to determine this contribution the effect of the boundary layer in the throat of the nozzle must be estimated. To achieve this the thermo-physical properties of the exhaust gasses at the throat must be recovered from the CEAM outputs of the nominal design (\autoref{subsec:nominal_sizing}). From them the throat Reynolds number can be obtained.
    \begin{equation}
        Re_t = \frac{\rho_t \, D_t \, u_t}{\mu_t}
    \end{equation}
Introducing the curvature of the throat, recovered from the Rao nozzle geometry, a modified Reynolds number is derived as follows:
    \begin{gather}
        k_t = 0.382 \, \frac{D_t}{2} \\
        Re'=\sqrt{\frac{D_t}{2 \, k_t}}Re_t
    \end{gather}  

    Now it is possible to calculate the throat discharge coefficient from which the real mass flow and effective throat area can be calculated.  

    \begin{gather}
        C_{d,t} = 1 - \left( \frac{\gamma_t+1}{2} \right)^{\frac{3}{4}}
        \left[3.266 - \frac{2.128}{\gamma_t+1} \right] \, \frac{1}{\sqrt{Re'}} + 0.9428 \frac{(\gamma_t - 1) \, (\gamma_t + 2)}{Re' \sqrt{\gamma_t + 1}} \\
        \dot{m}_r = C_{d,t} \, \dot{m}_{id} \\
        A_{t,eff} = \frac{\dot{m}_r \, c^*}{p_c}
    \end{gather}
\end{itemize}

\subsection{Effects on the nominal design}  

Including all the losses in the dynamic model reduces the overall performance of the system as expected, however, they only have a minor effect on the evolution of the throat area and its discharge coefficient as can be seen both in \autoref{fig:LS_At} and \autoref{fig:LS_Cd}. This is mainly due to the fact that the erosion rate was recovered by imposing a rather small variation of the throat area over time. 

\twofig{LS_At}{Throat area evolution}{LS_Cd}{Throat discharge coefficient evolution}{1}

The real throat area describes its physical evolution due to erosion while the apparent one is the effective area as seen by the gas flow due also to the presence of the boundary layer. The values found for the discharge coefficients, instead, are compatible with values found in literature together with the range of the modified Reynolds number used to compute them. \cite{slides_maggi} These variations have repercussions over the entire system as the chamber pressure evolution is influenced, so are the pressure cascade, the propellants mass flow rates and all consequent physical quantities as can be seen in the following figures.  

\twofig{LS_T}{Thrust comparison}{LS_Isp}{Specific impulse comparison}{1}
\twofig{LS_m}{Propellants mass flow rate comparison}{LS_pc}{Combustion chamber pressure comparison}{1}

Again it is possible to appreciate the fairly small repercussions of these losses as all quantities exhibit basically the same behavior except the specific impulse, which shows a larger separation from the ideal case. The total impulse and the burn time also decrease slightly as reported in \autoref{table:I_tot_t_b_comp}.

\begin{table}[H]
    \renewcommand{\arraystretch}{1.25}
    \centering
    \begin{tabular}{|c|c|c|}
        \hline
        & $\boldsymbol{I_{tot} \; [\textbf{N} \, \textbf{s}]}$ & $\boldsymbol{t_b \; [\textbf{s}]}$ \\
        \hline
        \hline
        \textbf{Ideal} & $2.173 \cdot 10^6$ & 3715 \\
        \hline
        \textbf{Real} & $2.124 \cdot 10^6$ & 3652 \\
        \hline
    \end{tabular}
    \caption{Total impulse and burn time comparison}
    \label{table:I_tot_t_b_comp}
\end{table}