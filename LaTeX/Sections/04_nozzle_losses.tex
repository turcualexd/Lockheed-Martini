\section{Nozzle losses}
\label{sec:nozzle_losses}

In order to calculate and evaluate the nozzle losses, further modifications and calculations were added to the model previously presented in \mref.
In particular 2D, throat erosion and boundary layer losses were considered.

\subsection{Losses calculations}  

Each loss term has been calculated as follows:
\begin{itemize}
    \item \textbf{2D losses}: for a parabolic Rao nozzle this loss can be computed in a similar way as a conical nozzle by applying \autoref{eq:2D_loss}
    \begin{equation}
        \lambda = \frac{1}{2} \left[ 1 + \cos \left( \frac{\delta + \theta_e}{2}\right)\right]
        \label{eq:2D_loss}
    \end{equation}
    where $\delta$ is the cone angle of an fictitious conical nozzle with the same divergent length and area ratio \cite{Sutton}.
    \item \textbf{Throat erosion losses:} this effect is due to the increasing throat area over time whose behavior can be obtained by considering a constant erosion rate for simplicity. Since this loss is time dependent it needs to be considered inside the dynamic model of the system (\mref).
    \begin{equation}
        D_t^{(i+1)} = D_t^{(i)} + 2 E_t \Delta t
    \end{equation}

    Usually the erosion rate is calculated through experimental measurements of the propulsion system, in this case a suitable erosion rate has been searched for in literature. Due to the smallness of the system, no acceptable rates were found, therefore, an increase of 2\% of the initial throat radius over the entire burn was assumed\cite{Sutton}\cite{tesi_malesia}. This gives an erosion rate of about \mref.
    \item \textbf{Boundary layer losses:} to determine this contribution the effect of the boundary layer in the throat of the nozzle must be estimated. To achieve this the thermophysical properties of the exhaust gasses at the throat must be recovered from the CEAM outputs of the nominal design (\mref). From them the throat Reynolds number can be obtained.
    \begin{equation}
        Re_t = \frac{\rho_t \, D_t \, u_t}{\mu_t}
    \end{equation}

    Introducing the curvature of the throat, recovered from the Rao nozzle geometry, a modified Reynolds number is derived as follows:
    \begin{gather}
        k_t = 0.382 \, \frac{D_t}{2} \\
        Re'=\sqrt{\frac{D_t}{2 \, k_t}}Re_t
    \end{gather}  

    Now it is possible to calculate the throat discharge coefficient from which the real mass flow and effective throat area can be calculated.  

    \begin{gather}
        C_{d,t} = 1 - \left( \frac{\gamma_t+1}{2} \right)^{\frac{3}{4}}
        \left[3.266 - \frac{2.128}{\gamma_t+1} \right] \, \frac{1}{\sqrt{Re'}} + 0.9428 \frac{(\gamma_t - 1) \, (\gamma_t + 2)}{Re' \sqrt{\gamma_t + 1}} \\
        \dot{m}_r = C_{d,t} \, \dot{m}_{id} \\
        A_{t,eff} = \frac{\dot{m}_r \, c^*}{p_c}
    \end{gather}
\end{itemize}

\subsection{Effects on the nominal design}  

To analyze better the Nozzle losses three different type of computation were made and compared to the ideal one, the first only consider the losses caused by the boundary layer, the second one consider only the throat erosion losses and the last one consider both. In all these cases the mass flow rate, the total impulse and the specific impulse have been calculated and compared.  

 

\subsubsection{Only boundary layer losses}  

%\twofig{cdcoeff}{Impulso specifico}{cdcoeff}{pressione c}{1}
[Immagini Is, T, mre] 

In this case only the boundary layer losses were considered. As expected, the real case is slightly less performing than the ideal case, as can be seen from the three figures comparing thrust, mass flow rate and the specific impulse. The total specific impulse was also calculated with a value of 2173.157 KNs for the ideal case and 2147.295 KNs for the real one. It should also be noted that the discharge coefficient is not stationary during the simulation, mainly because Dry-1 is a blowdown propulsion system where the mass flow in the nozzle is not constant during the burning time. This effect also affects the size of the effective throat area through which the flow passes. Lastly, is possible to observe that in the non-ideal case the burning time is slightly higher than in the ideal one, this happen because the real mass flow rate is decrease from the ideal condition an so the total mass of propellant is consumed slowly. The burning time for the ideal case is 3716 s and for the non-ideal one is 3760 s.             

[immagini Cd e Ateffective]  
     
\subsubsection{Nozzle losses only throat erosion} 

[Immagini Is, T, mre]      

The same parameters used in the previous case were analyzed in the throat erosion analysis. As expected, the specific and total impulses are lower for the non-ideal case compared to the ideal one. The total impulse is 2173.157 KNs for the ideal case and 2159.229 KNs for the losses case. However, for thrust and mass flow rate, the non-ideal case has a better performance than the ideal one. This is acceptable because the throat radius increases during the propulsion phase and more propellant can pass through, thus increasing the thrust and mass flow rate. Furthermore, it can be seen that the total combustion time for the non-ideal case is lower compared to the ideal one, with a total time of 3621 s for the first and 3716 s for the second. This is easily explained by the fact that an increasing throat allows more fuel to be expelled in less time and the pressure chamber to be reduced at a faster rate. Graphs of the changes in throat area and pressure chamber in both the ideal and non-ideal cases have been added for a better understanding of the phenomenon. 

Scrivere valore ts!!!!! 

[immagini At e Pc]

\subsubsection{Complete nozzle losses}  

[immagini Is, T, mre]   

Also, in this case the real case is not performing well as the ideal one. The result is a combination of both the erosion losses and boundary layer losses. It is possible to note that boundary layer losses are more prevalent in the beginning of the simulation, instead at the end the erosion one becomes visible. This is due to the fact that the nozzle erosion effects are more relevant when the throat has increased a lot. For this case the total impulse is 2173.157 KNs for the ideal case and 2139.825 KNs for the non-ideal one. By using the same justifications done in the previous section the difference in burning time can be justified. As in the only throat erosion case, also in this case there is a small difference in the burning times, with a 3716 s for the ideal one and 3679 s for the non-ideal, the same justification can be done. 

[immagini grafici]       