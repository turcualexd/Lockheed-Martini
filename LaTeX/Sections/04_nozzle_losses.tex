\section{Nozzle losses}
\label{sec:nozzle_losses}

In order to calculate and evaluate the nozzle losses, further modifications and calculations were added to the model previously presented in \mref.
In particular 2D, throat erosion and boundary layer losses were considered.

\subsection{Losses calculations}  

Each loss term has been calculated as follows:
\begin{itemize}
    \item \textbf{2D losses}: for a parabolic Rao nozzle this loss can be computed in a similar way as a conical nozzle by applying \autoref{eq:2D_loss}
    \begin{equation}
        \lambda = \frac{1}{2} \left[ 1 + \cos \left( \frac{\delta + \theta_e}{2}\right)\right]
        \label{eq:2D_loss}
    \end{equation}
    where $\delta$ is the cone angle of an fictitious conical nozzle with the same divergent length and area ratio \cite{Sutton}.
    \item \textbf{Throat erosion losses:} this effect is due to the increasing throat area over time whose behavior can be obtained by considering a constant erosion rate for simplicity. Since this loss is time dependent it needs to be considered inside the dynamic model of the system (\mref).
    \begin{equation}
        D_t^{(i+1)} = D_t^{(i)} + 2 E_t \Delta t
    \end{equation}

    Usually the erosion rate is calculated through experimental measurements of the propulsion system, in this case a suitable erosion rate has been searched for in literature. Due to the smallness of the system, no acceptable rates were found, therefore, an increase of 2\% of the initial throat radius over the entire burn was assumed\cite{Sutton}\cite{tesi_malesia}. This gives an erosion rate of about $3.189*10^{-2}$ $\frac{\mu*m}{s}$ \mref.
    \item \textbf{Boundary layer losses:} to determine this contribution the effect of the boundary layer in the throat of the nozzle must be estimated, this was done by using the . To achieve this the thermophysical properties of the exhaust gasses at the throat must be recovered from the CEAM outputs of the nominal design (\mref). From them the throat Reynolds number can be obtained.
    \begin{equation}
        Re_t = \frac{\rho_t \, D_t \, u_t}{\mu_t}
    \end{equation}

    Introducing the curvature of the throat, recovered from the Rao nozzle geometry, a modified Reynolds number is derived as follows:
    \begin{gather}
        k_t = 0.382 \, \frac{D_t}{2} \\
        Re'=\sqrt{\frac{D_t}{2 \, k_t}}Re_t
    \end{gather}  

    Now it is possible to calculate the throat discharge coefficient from which the real mass flow and effective throat area can be calculated.  

    \begin{gather}
        C_{d,t} = 1 - \left( \frac{\gamma_t+1}{2} \right)^{\frac{3}{4}}
        \left[3.266 - \frac{2.128}{\gamma_t+1} \right] \, \frac{1}{\sqrt{Re'}} + 0.9428 \frac{(\gamma_t - 1) \, (\gamma_t + 2)}{Re' \sqrt{\gamma_t + 1}} \\
        \dot{m}_r = C_{d,t} \, \dot{m}_{id} \\
        A_{t,eff} = \frac{\dot{m}_r \, c^*}{p_c}
    \end{gather}
\end{itemize}

\subsection{Effects on the nominal design}  

Including all the losses in the dynamic model has a significant effect on the evolution of the throat area as can be seen in \mref.


As expected the result of the simulation is that the case that consider nozzle losses is less performing respect to the ideal case. This is supported by analyzing the total impulse values, that are 2173.157 KNs for the ideal case and 2139.825 KNs for the non-ideal one. Moreover, it is possible to note that boundary layer losses are more prevalent in the in the simulation, this is totally expected because throat erosion losses typically do not affect much the performance of liquid propulsion system.  Futhermore, it can be noted that there is a small difference in the burning time of the two simulation, with a 3716 s for the ideal case and 3679 s for the non-ideal one. The cause of this effect can be attributed to the presence of the throat erosion that allow to discharge all the propellant quicker. Lastely, it is also worth noting that the values found for the discharge coefficients are compatible with the value found in the literature \mref [fig ,presa da slide maggi]with similar modified Reynolds number, the value range are respectably $0.9804-0.9697$ and $1.48*10^4-6.15*10^3$.    

[immagini grafici]