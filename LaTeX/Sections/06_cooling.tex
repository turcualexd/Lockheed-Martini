\section{Cooling analysis}
\label{sec:cooling}

In this section a preliminary feasibility analysis for cooling of the DRY-1 engine with RP-1 is discussed.
High temperatures reached in the combustion chamber and nozzle have to be deeply discussed at this first stage of the design since it could have deep consequences on the architecture, materials choice and performances of the engine. The following analysis is necessary to evaluate the researched feasibility of RP-1 as a coolant. In this context, it is possible to make reference to past successful engine design since kerosene cooling has been of fundamental importance for vast applications.

Two kinds of techniques could be considered with this propellant, regenerative cooling and film cooling. The last one is used along the lateral surface of the combustion chamber to create a fuel layer that protects the chamber, with a slight decrease on the performance. This strategy has demonstrated great effectiveness for in-space and low pressure engines \cite{cambridge}, an example of this is given by the previously cited 400-N Liquid Rocket Engine for in-space applications \cite{ariane_datasheet}. DRY-1 could exploit film cooling as a viable option thats is also enhanced by previous successful design for this small-thrust kind of motor. Clearly, the feasibility shall be quantitatively assessed.

On the other way, regenerative cooling techniques are not frequently seen on small thrust or low pressure chamber pressure applications since the critical point of the process relies on having sufficient pressure for the fuel line that is affected by losses along the cooling jacket. Considering the DRY-1 blow-down architecture, the critical points would be:

\begin{itemize}
    \item \textbf{Fuel feeding line pressure losses}: the pressure in the lines of feeding is varying along the operation, in particular at the end of the mission low fuel pressure could not match the requirement of pressure at the combustion chamber.
    \item \textbf{Variation of boiling temperature}: due to variation of fuel feeding pressure, also the vaporization temperature varies. In general temperature shall be assessed through the pipes to evaluate if decomposition or vaporization of RP-1 is happening. The mathematical discussion of the feasibility will be evaluated in \autoref{subsec:math_cooling}.
\end{itemize}

\subsection{Mathematical model for the gas heat transfer}
\label{subsec:math_cooling}

The workflow of this section is to evaluate whether the total heat flux generated by the gas flow (as convective heat) can be absorbed by the coolant without exceeding temperature constraint of the RP-1. Different approaches and assumptions can be made to breakdown the problem. In this case it is assumed to have a one-dimensional flow (perpendicular to the expanded gas flow), the heat is exchanged only by forced convective processes (disregarding the radiation coming from the flow), steady state assumption is made. Some other considerations will be presented during the following development.
Regarding the model of convective heat transfer from the hot gases to the wall, it can be mathematically described as:

\begin{equation}
    \dot{q} = h_g (T_{aw} - T_{wg})
\end{equation}

$T_{wg}$ is the static temperature of the wall on the internal side. The temperature considered for the gas flow $T_{aw}$ is referred as the total adiabatic wall temperature and is calculated as:

\begin{equation}
    T_{aw} = T_c \left[\frac{1 + r\left(\frac{\gamma - 1}{2}\right)M^2}{1 + \left(\frac{\gamma - 1}{2}\right)M^2}\right] \qquad r = Pr^{0.33}
\end{equation}

\vspace*{3mm}
Where Mach number is the one relative to the axial distance, in which $T_{wg}$ has to be calculated, the recovery factor $r$ depends on the local Prandtl number. Also $\gamma$ is assessed at axial coordinate of interest.
$T_{aw}$ is the temperature driving the heat transfer and it is slightly less then stagnation temperature of the gas due to the back heat effect generated by the compressible and viscous boundary layer near the chamber wall. The difficulty relies on establishing the value of the convective heat transfer $h_g$, different empirical models are present in the literature. One of the most used and commonly applied is the Bartz relation given as:


\begin{gather}
    h_g = \left(\frac{0.026}{D_t^2}\right) \left(\frac{\mu^{0.2} c_P}{Pr^{0.6}}\right)_{c} \left(\frac{p_c}{c^*}\right)^{0.8} \left(\frac{D_t}{k_t} \right)^{0.1} \left(\frac{A_t}{A}\right)^{0.9} \sigma
    \label{eq:h_g}
    \\[5pt]
    \sigma = \frac{1}{\left[\frac{1}{2}\frac{T_{wg}}{T_c} \left( 1 + \frac{\gamma + 1}{2}M^2 +\frac{1}{2} \right)\right]^{0.68} \left[1 + \frac{\gamma - 1}{2}M^2 \right]^{0.12}}
    \label{eq:sigma}
\end{gather}

\vspace{3mm}
Regarding \autoref{eq:h_g}, it depends on the geometry of the nozzle and chamber conditions. The area ratio of the same formula gives the dependence on the axial distance. The factor $\sigma$ in \autoref{eq:sigma} is also dependent on the axial distance of the nozzle since it contains the Mach number and the temperature of the wall on the gas side at that location. In the same equation, $\gamma$ is also referred to the local value at that position. It can be noticed that the unknown temperature $T_{wg}$ is present also in the $h_g$ coefficient: an assumption on the wall temperature must be taken. In particular, when the material of the nozzle wall is defined, also a maximum allowed temperature is provided. A reasonable assumption for the adiabatic temperature distribution  $T_{wg}$ can be done assuming that the ratio  $T_{wg} (x) / T_{aw} (x)$ is constant and equal to the same ratio evaluated at the nozzle inlet, imposing $T_{wg} = T_{max}$ at the nozzle entrance.
In \autoref{fig:CL_Twg} it is represented the distribution of the assumed $T_{wg}$ along the nozzle.

\cfig{CL_Twg}{Assume distribution for the wall temperature at the gas side}{0.4}

The maximum allowed temperature for the wall side is at the inlet of the nozzle. The material selected is Nimonic Alloy 75, a nickel-chromium alloy with maximum allowed temperature of 1600K \cite{nimonic}. Hence, this value was set as the $T_{wg}$ at the nozzle inlet. A more conservative approach should consider also lower values of the maximum reachable wall temperature to account for the decrease of material mechanical properties.

In this way, from the calculation of $h_g$ also the heat flux along the nozzle can be evaluated, from the convergent to the exit section.
The total heat rate can be estimated by integrating the heat flux over the internal area of the nozzle. This value can be found by numerical integration and the use of the Rao nozzle geometry introduced in \autoref{subsec:initial_sizing}. Finally, the increase of temperature of the fuel can be found by assuming the specific heat of the liquid.

\begin{gather}
    \dot{Q} = 2 \pi \int_{0}^{L_{tot}} \dot{q} y \; dx
    \\
    \Delta T_{fu} = \frac{\dot{Q}}{c_{fu} \dot{m}_{fu}}
\end{gather}

At this point, another consideration has to be pointed out: the condition that was set up by imposing the gas side wall temperature at its maximum (depending on the material choice) is a conservative choice since at the steady state it is related to the minimum flow that will be absorbed by the coolant (assuming fixed stationary gas flow conditions). If the coolant would not withstand this condition, the regenerative cooling could be excluded or eventually some other material or coating shall be used for the nozzle.

The heat transfer evaluation has also been combined with the system discharge dynamics (considering $c^*$ as constant as commented in \autoref{subsec:dynamics}) in order to evaluate the heat flux rate, the heat rate and the increase of temperature for the fuel at every instant.
The data for the gas properties can be found in the same simulation with CEAM. The considered specific heat for RP-1 was $c = 1800$ J/kgK.
The quasi-static condition is assumed since the steady-state heat transfer is considered at every time instant. Also the wall temperature distribution is considered unchanged in time.
% aggiungere che non riflette esattamente la realta poiche sto considerando sempre la massima temperatura sul muro e non sto considerando transienti

\begin{table}[H]
    \renewcommand{\arraystretch}{1.5}
    \centering
    \begin{tabular}{|c|c|c|}
        \hline
        $\boldsymbol{\Delta T_{co,1} \; [\textbf{K}]}$ & $\boldsymbol{\Delta T_{co,2} \; [\textbf{K}]}$ & $\boldsymbol{\Delta T_{co,3} \; [\textbf{K}]}$ \\
        \hline
        \hline
        804.13 & 987.61 & 990.53 \\
        \hline
    \end{tabular}
    \caption{Coolant increase of temperature considering three time instants}
    \label{table:dt_cooling}
\end{table}

\cfig{CL_q}{Heat flux along the nozzle at different time instants}{0.45}

\pagebreak
From \autoref{table:dt_cooling} it can be understood that the mass flow rate of the fuel is not enough to absorb the heat without having several effects. Two opposite effects happen during the discharge of the blow-down: the decrease of the convective heat transfer coefficient related to the chamber pressure, but also the decrease of the propellant mass flow rate. The latter effect is prevailing since the increase of $\Delta T_{co}$ is larger at later time instants. Such high values of $\Delta T_{co}$ would bring to vaporize the fuel considering that the storing temperature of the RP-1 is 300 K (boiling temperature for RP-1 happens in the range 623K - 798K \cite{rp_1_temp}). At high temperature decomposition process for the RP-1 can take place. Regarding the nozzle cooling, the radiative capability of the external wall shall be computed, also some other kinds of material or TBC for the internal wall such as $ZrO_2$  should be investigated. Summing up the result of the present investigation, it can be stated that regenerative cooling for such a small-sized engine as it is the DRY-1 could be excluded due to three critical aspects:
\begin{itemize}
    \item High heat fluxes due to the miniaturization $h_g \propto D^{-2.6}$. Considering motor with the same combustion initial conditions (same couple, $P_c$ and $O/F$), smaller engines will have to face more critical heating conditions;
    \item reduced mass flow rate of the fuel induces high $\Delta T_{co}$ to ensure power conservation;
    \item blow-down dynamics creates different conditions throughout the burn.
\end{itemize}

Concerning the cooling of the thrust chamber, a viable option could be the introduction of a film cooling layer across the internal wall of the combustion chamber. The usage of hydrocarbon-based coolant with this technique revealed effective results. Under pressure of 130 bar, the hydrocarbons deposit acts as a thermal insulator hence protecting the walls of the chamber \cite{huzel_huang}. In addition, the design of the film cooling should consider the re-design of the injector plate and should be evaluated in parallel with the material choice for the combustion chamber.