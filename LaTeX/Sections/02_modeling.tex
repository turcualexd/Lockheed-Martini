\section{Modeling of propulsion system: DRY-1}
\label{sec:modeling}

\pagebreak
The workflow for the sizing of DRY-1 is introduced by \autoref{fig:flowchart} and it is divided into three stages:
\begin{itemize}
    \item \textbf{Input data}:  the problem is set up. 
    \item \textbf{Nominal sizing}: the system is sized according to initial conditions and general assumptions.
    \item \textbf{Dynamics}: an iterative process is set up to model the blow-down dynamics and finalize the sizing.
\end{itemize}

\cfig{flowchart}{Flowchart of the sizing}{1}

\subsection{Input data}
\label{subsec:input}
The input data defined different kind of requirements, related to operability environment, engine performance, size constraints, chemistry, architecture and manufacturing techniques. Regardless of the development of the engine design, the system shall respect the following pinpoints:
\begin{itemize}
    \item \textbf{Environment}: vacuum for the whole operation.
    \item \textbf{Thrust}: initial magnitude of 1 kN, no lower boundary.
    \item \textbf{Chamber Pressure}: initial value of 50 bar, always above 20 bar throughout the whole mission.
    \item \textbf{Allocated Space}: tanks, combustion chamber and convergent nozzle occupancy is exactly 80\% of the volume occupied by a cylinder of 1 meter diameter and 2 meter length. No bounds on the extension of the divergent.
    \item \textbf{Propellants}: semi-cryogenic couple of LOX and RP-1.
    \item \textbf{Architecture}: blow-down type.
    \item \textbf{Manufacturing}: all the system is produced in AM, no restriction on material nor techniques. 
\end{itemize} 

The nominal sizing refers to the design of the overall system considering the imposed initial constraints as static conditions. This design choice was dictated by the dynamics of the blow-down system, which imposes the maximum flow rate at the beginning of the mission, leading to the oversizing of the engine throughout the rest of the mission. % da rivedere questa frase

Various hypothesis were necessary to develop the system, this values are reported in \autoref{table:hp}.

\begin{table}[H]
    \renewcommand{\arraystretch}{1.2}
    \centering
    \begin{tabular}{|c|c|c|c|}
        \hline
        $\boldsymbol{O/F} \; [-]$ & $\boldsymbol{\epsilon} \; [-]$ & $\boldsymbol{\epsilon_{con}} \; [-]$ & $\boldsymbol{L^*}$ \; [m] \\
        \hline
        2.42 & 300 & 10 & 1.143 \\
        \hline
    \end{tabular}
    \caption{Hypothesis from literature and previous design}
    \label{table:hp}
\end{table}

The choices of $\boldsymbol{L^*}$  and $\boldsymbol{O/F}$ were only dictated by the propellant couple \cite{sutton}, while $\boldsymbol{\epsilon}$ was chosen as the characteristic of the engine refers to an in-space application \cite{ariane_datasheet}. Regarding the value of $\boldsymbol{\epsilon_{con}}$ a mean value between 5 and 15 was taken. Smaller values entails longer combustion chamber and small cross sectional area, with large pressure drops. Larger values refers to bigger chamber cross sectional area, with limited length for the combustion. From the literature the suggestion for the choice of this value is to refer to previous successful engines design, considering the same application \cite{huzel_huang}. Therefore, a 400N Bi-Propellant apogee motor was taken as reference and revealed a value of $\boldsymbol{\epsilon_{con}} \approx$ 10 \cite{ariane_datasheet}.

\subsection{Nominal sizing}
\label{subsec:nominal}
After defining the main input data, the workflow is carried out as shown in \autoref{fig:flowchart}. All the combustion simulations were performed with Nasa-CEA software, implemented in Matlab (CEAM). In particular the "rocket problem" was considered, imposing frozen equilibrium, infinite combustion chamber ($\boldsymbol{M_c}$ = 0), initial injecting temperatures of the propellant equal to the storage temperatures. The chamber pressure was set as $\boldsymbol{P_c}$ = 50 bar and the mixture ratio as $O/F$ = 2.42. Latter refinement of this last value will be performed. The output values used from the simulation are represented (vacuum value is considered for the $\boldsymbol{c_T}$):
\begin{table}[H]
    \renewcommand{\arraystretch}{1.2}
    \centering
    \begin{tabular}{|c|c|c|}
        \hline
        $\boldsymbol{c^*} \; [m/s]$ & $\boldsymbol{c_T} \; [-]$ & $\boldsymbol{T_c} \; [K]$ \\
        \hline
        1851 & 1.935 & 3709  \\
        \hline
    \end{tabular}
    \caption{First run on CEAM}
    \label{table:out_CEA_1}
\end{table}

%equazioni per mass flow rate e aree motore + risultati solo motore

%equazioni tank --> risultati per tank 

%equazioni feeding --> risultati feeding



\subsection{System dynamics}
\label{subsec:dynamics}

