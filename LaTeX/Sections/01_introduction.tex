\section{Introduction and literature overview}
\label{sec:introduction}

\subsection{Blow-down heritage}
\label{subsec:blowdown_intro}

\subsection{Additive manufacturing state of art}
\label{subsec:additive_intro}



\subsection{Analysis of losses}
\label{subsec:losses_intro}
 The model of the system was based on some ideal assumption of the propulsion process, but in the nozzle some irreversible processes and losses are present. Therefore some of them were analyzed and compared with the ideal case to achieve a better understanding of what really happens in the nozzle. The losses considered are specifically the ones caused by 2D flow, throat erosion and boundary layer. Considering 2D flow means that the propellants exit velocity is slightly misaligned from the nozzle axis, especially near its edges. Therefore only part of the flow will contribute to the thrust of the engine. The throat erosion is mostly caused by the exhaust gasses passing through it with high velocity and temperature, thus causing the material of the nozzle to erode and fail more easily. This effect causes an unwanted expansion of the throat area during the mission consequently increasing the rate at which the combustion chamber pressure decreases over time. The boundary layer loss is caused by the presence of a boundary layer between the nozzle wall and the flowing gasses. It is present along the whole nozzle but its effects are particularly evident in the throat, as it is the smallest part of the nozzle. This effect is accentuated even more by general dimensions of the considered system. 

