\section{Introduction and literature overview}
\label{sec:introduction}

\subsection{Blow-down heritage}
\label{subsec:blowdown_intro}

\subsection{Additive manufacturing state of art}
\label{subsec:additive_intro}

%traditional manifacturing problems->additive advantage (tyempi, costi, peso, geometrie complesse->ottimizzate)
Additive manufacturing technologies in aerospace industry is rapidly spreading thanks to the offered advantages with respect to traditional production techniques. Among the benefits provided by AM there are weight and time savings, cost reductions and  the possibility to optimize components creating more complex geometries\cite{materials_and_desing}.
Various technologies for AM have been developed for different fields, in particular, the most used in this sector is Laser Powder Bed Fusion (LPBF), which utilizes a laser beam to melt successive layers of metal powder. Different types of alloys can be processed through this technology, among which there are nickel-based superalloys, such as Inconel-718. This materials offers excellent mechanical properties under extreme temperature conditions and, for this precise reason, suits perfectly the role of injectors, manifolds and turbomachinery elements in general. Inconel-718 is also characterized by a considerable corrosion resistance, that allows it to be used in concomitance with an oxidizer such as liquid oxygen\cite{Inconel_vs_steel}. The finishing of a component created with AM depends on several parameters: 
\begin{itemize}
    \item \textbf{Temperature}: the laser's power and thus the powder's melting temperature selection is crucial in order to avoid performance losses due to defects in melting and cooling. The calculation of this parameter is important to prevent overflowing from the meltpool, convective transport of the liquid and nonmetal particles to be blown away from vapors. Cracks and curling, caused by uneven shrinkage during production can also damage surface quality\cite{lpbf_accuracy}. 
    \item  \textbf{Powder dimension and morphology}: the mechanical properties of the component are significantly influenced by the dimensions and shape of the powder. Coarse and irregular particles result in inefficient packing and higher porosity of the powder bed. This, in turn, makes it challenging for the molten material to fill all the interstices, especially due to the rapid solidification in the LPBF process. The mechanical properties of the component decrease due to its porosity\cite{dimension_powder}.
    \item  \textbf{Deposition method}: powder deposition direction directly affects the superficial finishing of the material as dross or deposits from the process can be present: the angle of deposition directly influences the roughness and the accuracy of the produced parts. Staircase effects is related to the discretization of the different layers: the bigger the layers the less smooth the surface will be. In general, it can be observed that the more the build direction $\alpha$ of a generic piece shifts from perpendicular ( $\alpha$ = 0°), to parallel, ( $\alpha$ = 90°), the roughness increases and accuracy lowers\cite{tesi_dottorato};
\end{itemize}

\subsection{Analysis of losses}
\label{subsec:losses_intro}
Real life is far from an ideal world, and this is also true for propulsion systems, especially around the nozzle. The modelling considered is based on some ideal assumptions of the propulsion process, but in the nozzle some irreversibilities and losses have been analyzed and compared with the ideal case. The losses considered are specifically those caused by throat erosion and boundary layer thickness\cite{Sutton}. The throat erosion of a nozzle is mainly caused by the fact that the propellant passes through the throat at a high velocity and at an elevated temperature, causing the material around it to erode and fail more easily. This effect causes an unwanted increase in the throat area of the nozzle during propulsion, increasing the rate at which the combustion chamber's pressure decreases and as a result, provoking a loss of power of the system. The second irreversibility analyzed is that caused by the viscosity between the propellant flow and the nozzle surface. As the nozzle is stationary and the flow is moving, a boundary layer is created which causes a slight decrease in the average exit velocity. This can be simplified by saying that the effective throat area through which the flow can pass is reduced, and therefore the actual flow rate through is not the ideal one, thus worsening the merit parameters\cite{slides_maggi}.