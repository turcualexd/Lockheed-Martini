\section{Introduction and literature overview}
\label{sec:introduction}

\subsection{Blow-down heritage}
\label{subsec:blowdown_intro}

\subsection{Additive manufacturing state of art}
\label{subsec:additive_intro}



\subsection{Analysis of losses}
\label{subsec:losses_intro}
Real life is far away from an ideal world, this is also true for propulsion systems, especially around the nozzle. The considered modelling is based on some ideal assumption of the propulsion process, but in the nozzle some irreversibility and losses were analyzed and compared with the ideal case. The losses considered are specifically the ones caused by throat erosion and displacement thickness. The throat erosion of a nozzle, is mostly caused by the propellant passing through the throat with a high velocity and with high temperature and thus causing the nozzle's material around the throat to erode and fail more easily. This effect causes an unwanted increases of the nozzle's throat area during the propulsion and thus increase the rate at which the combustion chamber pressure decreases. Therefore, the propulsion system loses in its performance. The second analyzed irreversibility is the one caused by the viscosity between the propellant's flow and the nozzle's surface. Due to the fact that the nozzle is stationary and the flow is moving a boudery layer is created causing a slitely decrease in the total       

